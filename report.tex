\documentclass[a4paper, 12pt]{article}

\usepackage[no-math]{fontspec-xetex}
\setmainfont{IBM Plex Sans}
\usepackage[english, russian]{babel}

\usepackage{blindtext}
\usepackage{microtype}
\usepackage{geometry}

\usepackage{amsmath, amsfonts, amssymb, amsthm, mathtools}
\usepackage{MnSymbol}
\usepackage{physics}

\usepackage{graphicx, wrapfig, caption, subcaption}
\usepackage{color, xcolor}

\usepackage{titlesec}
\usepackage[document]{ragged2e}
\usepackage{enumitem}
%\usepackage{csvsimple}
%\usepackage[none]{hyphenat}

\graphicspath{{figures/}}
\geometry{margin=6em}
\geometry{bottom=6em}

% Настройки заголовков
\titleformat{\section}[hang]{\Large}{}{0em}{}{}
\titleformat{\subsection}[hang]{\large}{}{0em}{}{}

% Это предотвратит разрыв слов
\hyphenpenalty=10000
\exhyphenpenalty=10000

% Настройки параграфов текста
\setlength{\parindent}{0em}
\setlength{\parskip}{0.5em}
\renewcommand{\baselinestretch}{1.1}

% Настройки списков
\setlist{leftmargin=*, noitemsep}

% Настройки таблиц
\setlength{\tabcolsep}{2em}
\renewcommand{\arraystretch}{1.0}

\title{Определение коэффициента вязкости глицерина}
\author{Роман Ухоботов, Николай Грузинов}
\date{}%собрано \today}

\begin{document}
    \maketitle


    \section{Цель работы}\label{sec:target}
    Определить коэффициент вязкости жидкости (глицерина) при помощи исследования падающих в него тел.


    \section{Используемое оборудование}\label{sec:tools}
    \begin{enumerate}
        \item Мерный цилиндр с глицерином
        \item Набор калиброванных шариков из стали и свинца
        \item Измерительные приборы --- весы, микрометр, линейка
        \item Оборудование для съёмки --- камера телефона и фонарь
    \end{enumerate}


    \section{Теория}\label{sec:theory}

    Для падения тел сферической формы в вязких жидкостях действует формула Стокса:
    \[F(v) = -6 \pi r \eta v\]

    Поэтому I закон Ньютона для шарика с установившейся скоростью выглядит следующим образом:
    \[m g = 6 \pi r \eta v + \rho g V\]

    Выразим из него вязкость глицерина:
    \[ \eta = \frac{m g - \rho g V}{6 \pi r v} \]

    А установившуюся скорость в свою очередь попробуем установить экспериментально.


    \section{Методика измерений}

    Мы решили, что точнее всего получится проанализировать видео падающего шарика с помощью алгоритма,
    чтобы установить его скорость и, соответственно, вязкость жидкости.
    Проделаем эксперимент 10 раз, предварительно замерив параметры подопытных шариков --- 3 стальных и 7 свинцовых:

    \vspace{1.5em}

    \begin{tabular}{l l l}
        \bfseries Number & \bfseries Diameter (mm) & \bfseries Mass (g) \\ \hline
        1                & 3.972                   & 0.2570             \\ \hline
        2                & 3.974                   & 0.2575             \\ \hline
        3                & 3.964                   & 0.2556             \\ \hline
        4                & 2.181                   & 0.0834             \\ \hline
        5                & 2.430                   & 0.0929             \\ \hline
        6                & 2.984                   & 0.1705             \\ \hline
        7                & 3.003                   & 0.1584             \\ \hline
        8                & 4.042                   & 0.3882             \\ \hline
        9                & 4.013                   & 0.3824             \\ \hline
        10               & 4.098                   & 0.3864             \\ \hline
    \end{tabular}


%    Как мы поняли, для получения вязкости нам необходимо определить установившуюся скорость шарика.
%    Для этого найдём зависимость скорости от времени падения.
%    В этом нам поможет II закон Ньютона для шарика в произвольный момент падения, вертикальная ось направлена вверх:
%    \[ ma = - 6 \pi \eta r v - \rho g \frac{\pi d^3}{6} + mg \]
%
%    Решив дифференциальное уравнение относительно скорости, мы получаем:
%    \[ \tau = \frac{m}{6\pi r\eta}\ln\left( \frac{mg - \rho gV - 6\pi r\eta v_0}{mg - \rho gV - 6\pi r\eta v} \right) \]
%
%    По этой формуле можно оценить сверху примерное время становления скорости --- например, когда

\end{document}
