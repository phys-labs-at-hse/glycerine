\documentclass[a4paper, 12pt]{article}

\input{/home/nick/Desktop/notetaking/used_packages_ru.tex}
\input{/home/nick/Desktop/notetaking/user_defined_commands_ru.sty}

\graphicspath{{figures/}}
\geometry{margin = 2 cm}
\geometry{bottom = 3 cm}

\title{Теор модель}
\author{Николай Грузинов}
\date{}%собрано \today}

\begin{document}
\maketitle

Вначале шарик падает с какой-то скоростью.
Как мы убедились, эта скорость больше той, которая должна установится, поэтому скорость будет постепенно замедлятся.
Можно считать, что под конец скорость уже установилась.

Из наших данных $x(t)$ можно извлечь установившуюся скорость $V$, вычислив угловой коэффициент прямой, проходящей через последние точки:
\begin{center}
\includegraphics[width=0.7\linewidth]{approximation_0.png}
\end{center}

По $V$ можно найти вязкость $\eta$.
Возьму уравнение из методички, и немного перепишу.
Пусть $d$ --- диаметр шарика.
\[ m g = 3 \pi d \eta V+\rho g \frac{\pi d^{3}}{6} \]
\[ \eta = \frac{m g-\rho g \frac{\pi d^{3}}{6} }{3 \pi d V} \]

\section*{Текст ниже про то, как это стоило бы делать, если бы не большая погрешность при численном дифференцировании}

Пусть диаметр шарика $d$, масса $m$, плотность глицерина $\rho$, вязкость $\eta$.
Второй закон Ньютона для шарика, в проекции на ось OX, направленную вниз ($v_x > 0$)
% $\dv{v}{t} < 0$
\[ m \dv{v_x}{t} = -3 \pi \eta d v_x - \rho g \frac{\pi d^3}6 + mg \]
\[ \dv{v_x}{t} = -\frac{3 \pi \eta d}m v_x - \frac{\rho g \pi d^3}{6 m} + g \]
\[ \dv{v_x}{t} = -\frac{3 \pi \eta d}m \left(v_x + \frac{\frac{\rho g \pi d^3}{6} - mg}{3 \pi \eta d}\right) \]
Обозначу $c = \frac{mg - \frac{\rho g \pi d^3}{6}}{3 \pi \eta d}$, $v_x' = v_x - c$, $k = \frac{3 \pi \eta d}m$.
Тогда $c > 0$, $k > 0$, и
\[ \dv{v_x'}{t} = -k v_x' \qquad\implies\qquad v_x'(t) = \bigl(v_{x_0} - c\bigr)\exp(-kt) \]
\[ v_x(t) = \bigl(v_{x_0} -  c\bigr)\exp(-kt) + c .\] 
\[
x(t) = \int_{0}^t v_x(t') \dd t' = \int_{0}^t \Bigl((v_{x_0} -  c)\exp(-kt') + c\Bigr) \dd t' =
	   \frac{(v_{x_0} -  c)}{-k}\exp(-kt') + ct + x_0
\]
Видно, что из $x(t)$ вытащить $k$ (и затем из $k$ выразить $\eta$) будет трудно, потребуется анализировать нелинейную регрессию.
Поэтому я предлагаю численно продифференцировать исходные данные $x(t)$, получить оттуда $v_x(t)$.
Это упростит задачу следующим образом.
Если считать, что в конце пути скорость установилась, то $c$ --- среднее из нескольких последних точек $v_x(t)$.
Тогда можно линеаризовать $v(t)$ следующим образом:
\[ \ln( v_x(t) - c ) = \ln(v_{x_0} -  c) - k t .\] 
Угловой коэффициент графика будет равен $-k$.

Чтобы получить $c$ более точно, можно попробовать поподбирать такие значения, при которых прологарифмированная зависимость будет лучше ложиться на прямую.
Из $c$ можно получить плотность глицерина $\rho$ и сравнить ее с табличными значениями.

При взгляде на данные видно, что ничего хорошего из этой идеи не выйдет, потому что качественно посчитать скорость в каждой точке не представляется возможным.
\begin{center}
\includegraphics[width=0.6\linewidth]{bad_numericall_differentiation.png}
\end{center}
\end{document}
